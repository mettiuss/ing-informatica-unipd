\section{Linearizzazione}
Le equazioni ottenute in \eqref{eq:primaEquazioneCardinale} e \eqref{eq:secondaEquazioneCardinale} non sono lineari e non possono essere risolte analiticamente.
Per questo motivo è utile determinare un modello lineare che approssimi il comportamento del sistema attorno ad un certo punto, nel nostro caso il volo rettilineo simmetrico uniforme.

Per fare questo vengono risolte le equazioni del moto in un punto di equilibrio (indicheremo tali soluzioni con $\bar{x}$), così da poter riscrivere le variabili di stato come piccole perturbazioni rispetto a tale punto (che indicheremo con $\widetilde{x}(t)$), come descritto in \cite{zampieri_dispensa_controlli}.

\begin{equation*}
    x(t) = \bar{x} + \tilde{x}(t)
\end{equation*}

\subsection{Punto di Equilibrio}
Il punto di equilibrio scelto è quello di volo rettilineo simmetrico uniforme, le cui ipotesi per un aeromobile sono:
\begin{sitemize}
    \item Velocità costante: $\dot{u}(t) = \dot{v}(t) = \dot{w}(t) = 0$.
    \item Velocità angolare nulla, quindi accelerazione angolare nulla: $\bar{p} = \bar{q} = \bar{r} = \dot{p}(t) = \dot{q}(t) = \dot{r}(t) = 0$.
    \item Ali livellate: $\bar{\phi} = 0$.
    \item Angolo di derapata nullo: $\beta(t) = 0$.
\end{sitemize}

Nel caso in cui l'angolo di derapata sia nullo, l'equazione \eqref{eq:velocitaWind} diventa:
\begin{equation*}
    \begin{bmatrix}
        \bar{u} \\
        \bar{v} \\
        \bar{w}
    \end{bmatrix} = |\bar{V}_{FRD}| \begin{bmatrix}
        \cos(\bar{\alpha}) \\
        0                  \\
        \sin(\bar{\alpha})
    \end{bmatrix}
\end{equation*}

\begin{sitemize}
    \item quindi: $\bar{v} = 0$.
\end{sitemize}

Inoltre, come descritto in \cite{bryson_control_spacecraft_aircraft}, le forze aerodinamiche possono essere riassunte nei tre termini $X, Y, Z$:
\begin{equation*}
    \begin{split}
        X(t) & = -F_{A_{x}}(t)\cos\alpha(t)                         \\
        Y(t) & = F_{A_y}(t)                                         \\
        Z(t) & = -F_{A_x}(t)\sin\alpha(t) - F_{A_z}(t)\cos\alpha(t)
    \end{split}
\end{equation*}

\begin{note}
    Al fine di raggiungere un sistema di equazioni in cui $\delta_t(t)$ sia mantenuta come variabile di controllo, si è deciso di non incorporare la forza di propulsione $\vec{T}(t)$ nei termini ($X(t)$ $Y(t)$ $Z(t)$), anche se questa è un'alternativa valida che viene mostrata nel dettaglio in \cite{smith_aircraft_flight_mechanics}.
\end{note}

\subsubsection{Variabili del Sistema}

È quindi possibile riscrivere le variabili del sistema come:
\begin{equation*}
    \setlength{\arraycolsep}{2.5em}
    \begin{array}{lll}
        u(t) = \bar{u} + \tilde{u}(t) & p(t) = \tilde{p}(t) & \phi(t) = \tilde{\phi}(t)                    \\
        v(t) = \tilde{v}(t)           & q(t) = \tilde{q}(t) & \theta(t) = \bar{\theta} + \tilde{\theta}(t) \\
        w(t) = \bar{w} + \tilde{w}(t) & r(t) = \tilde{r}(t) & \psi(t) = \bar{\psi} + \tilde{\psi}(t)       \\
    \end{array}
\end{equation*}

\begin{equation*}
    \setlength{\arraycolsep}{1.5em}
    \begin{array}{lll}
        X(t) = \bar{X} + \tilde{X}(t) & L(t) = \bar{L} + \tilde{L}(t) &                               \\
        Y(t) = \bar{Y} + \tilde{Y}(t) & M(t) = \bar{M} + \tilde{M}(t) & T(t) = \bar{T} + \tilde{T}(t) \\
        Z(t) = \bar{Z} + \tilde{Z}(t) & N(t) = \bar{N} + \tilde{N}(t) &                               \\
    \end{array}
\end{equation*}

Ricordando il teorema di Taylor-McLaurin è possibile ricavare le seguenti approssimazioni per $\sin$ e $\cos$ per gli angoli di Eulero, tali espressioni saranno utili successivamente:

\begin{equation}
    \label{eq:EuleroSinCos}
    \begin{split}
        \sin(\bar{\theta} + \widetilde{\theta}(t)) & \approx \sin(\bar{\theta}) + \widetilde{\theta}(t)\cos(\bar{\theta}) \\
        \cos(\bar{\theta} + \widetilde{\theta}(t)) & \approx \cos(\bar{\theta}) - \widetilde{\theta}(t)\sin(\bar{\theta}) \\
        \sin(\widetilde{\phi}(t))                  & \approx \widetilde{\phi}(t)                                          \\
        \cos(\widetilde{\phi}(t))                  & \approx 1
    \end{split}
\end{equation}

\subsubsection{Equazioni Cardinali del Moto all'Equilibrio}

Sotto le ipotesi di volo rettilineo simmetrico le equazioni cardinali del moto all'equilibrio sono:

\begin{equation}
    \label{eq:equilibrio}
    \begin{split}
        \begin{bmatrix}
            0 \\
            0 \\
            0
        \end{bmatrix} & = mg\begin{bmatrix}
                                -sin\bar{\theta} \\
                                0                \\
                                cos\bar{\theta}
                            \end{bmatrix} + \begin{bmatrix}
                                                \bar{X} \\
                                                \bar{Y} \\
                                                \bar{Z}
                                            \end{bmatrix} + \begin{bmatrix}
                                                                \bar{T} \cos\epsilon \\
                                                                0                    \\
                                                                -\bar{T} \sin\epsilon
                                                            \end{bmatrix}
        \Rightarrow \begin{bmatrix}
                        \bar{X} \\
                        \bar{Y} \\
                        \bar{Z}
                    \end{bmatrix} = \begin{bmatrix}
                                        mg\sin\bar{\theta} + \bar{T} \cos\epsilon \\
                                        0                                         \\
                                        -mg\cos\bar{\theta} - \bar{T} \sin\epsilon
                                    \end{bmatrix}
        \\
        \begin{bmatrix}
            \bar{L} \\
            \bar{M} \\
            \bar{N}
        \end{bmatrix} & = \begin{bmatrix}
                              0 \\
                              0 \\
                              0
                          \end{bmatrix}
    \end{split}
\end{equation}

\subsection{Linearizzazione delle Equazioni}
A questo punto è possibile sostituire nelle equazioni del moto le variabili di stato come piccole perturbazioni rispetto al punto di equilibrio, come descritto in \cite{zampieri_dispensa_controlli}.

Per semplificare le equazioni, vengono utilizzate le relazioni trovate in \eqref{eq:EuleroSinCos} e si trascurano i termini di ordine superiore al primo in quanto le perturbazioni sono piccole ($\widetilde{x} \rightarrow 0$).

\subsubsection{Linearizzazione della Velocità Angolare}
Dall'equazione \eqref{eq:EulerotoVelocitaAngolare} che lega la velocità angolare con le derivate degli angoli di Eulero si ottiene la seguente relazione lineare:

\begin{equation}
    \label{eq:EulerotoVelocitaAngolareLineare}
    \begin{split}
        \begin{bmatrix}
            \widetilde{p}(t) \\
            \widetilde{q}(t) \\
            \widetilde{r}(t)
        \end{bmatrix} & = \begin{bmatrix}
                              1 & 0                          & -\sin(\bar{\theta} + \widetilde{\theta}(t))                         \\
                              0 & \cos(\widetilde{\phi}(t))  & \sin(\widetilde{\phi}(t))\cos(\bar{\theta} + \widetilde{\theta}(t)) \\
                              0 & -\sin(\widetilde{\phi}(t)) & \cos(\widetilde{\phi}(t))\cos(\bar{\theta} + \widetilde{\theta}(t))
                          \end{bmatrix}\begin{bmatrix}
                                           \dot{\widetilde{\phi}}(t)   \\
                                           \dot{\widetilde{\theta}}(t) \\
                                           \dot{\widetilde{\psi}}(t)
                                       \end{bmatrix} \\ &= \begin{bmatrix}
            1 & 0 & -\sin\bar{\theta} \\
            0 & 1 & 0                 \\
            0 & 0 & \cos\bar{\theta}
        \end{bmatrix}\begin{bmatrix}
            \dot{\widetilde{\phi}}(t)   \\
            \dot{\widetilde{\theta}}(t) \\
            \dot{\widetilde{\psi}}(t)
        \end{bmatrix}
    \end{split}
\end{equation}
\subsubsection{Linearizzazione delle Equazioni Cardinali}
Dalle equazioni cardinali si ottengono le seguenti relazioni lineari:

\begin{equation}
    \label{eq:primaEquazioneCardinaleLineare}
    \begin{split}
        m\left(\dot{\widetilde{u}}(t) + \widetilde{q}(t)\bar{w}\right)                           & = -mg\widetilde{\theta}(t)\cos\bar{\theta} + \widetilde{X}(t) + \widetilde{T}(t)\cos\epsilon \\
        m\left(\dot{\widetilde{v}}(t) + \widetilde{r}(t)\bar{u} - \widetilde{p}(t)\bar{w}\right) & = mg\cos(\bar{\theta})\widetilde{\phi}(t) + \widetilde{Y}(t)                                 \\
        m\left(\dot{\widetilde{w}}(t) - \widetilde{q}(t)\bar{u}\right)                           & = -mg\widetilde{\theta}(t)\sin\bar{\theta} + \widetilde{Z}(t) - \widetilde{T}(t)\sin\epsilon
    \end{split}
\end{equation}

\begin{equation}
    \label{eq:secondaEquazioneCardinaleLineare}
    \begin{split}
        \widetilde{L}(t) & = \dot{\widetilde{p}}(t)I_{xx} - \dot{\widetilde{r}}(t)I_{xz} \\
        \widetilde{M}(t) & = \dot{\widetilde{q}}(t)I_{yy}                                \\
        \widetilde{N}(t) & = \dot{\widetilde{r}}(t)I_{zz} - \dot{\widetilde{p}}(t)I_{xz}
    \end{split}
\end{equation}

\subsubsection{Linearizzazione Angolo di Derapata}

Dall'equazione \eqref{eq:velocitaWind} e ipotizzando $v^2, w^2 \ll u^2$, come descritto in \cite{bryson_control_spacecraft_aircraft}, si ottiene la relazione:

\begin{equation}
    \label{eq:angoloDerapataLineare}
    \beta(t) = \arcsin \frac{v(t)}{\left|\vec{V}_{FRD}(t)\right| } = \arcsin \frac{\widetilde{v}(t)}{\sqrt{u^2(t) + v^2(t) + w^2(t)}} \approx \frac{\widetilde{v}(t)}{\bar{u}}
\end{equation}

\subsubsection{Linearizzazione della Velocità Assoluta}
Dall'equazione \eqref{eq:velocitaNED} si ottiene la seguente equazione per la componente della velocità lungo l'asse $\hat{z}_{NED}$:

\begin{equation}
    \label{eq:velocitaNEDZLineare}
    \dot{\widetilde{z}}(t) = -\sin\bar{\theta}(\bar{u} + \widetilde{u}(t)) - \widetilde{\theta}(t)\bar{u} + \cos\bar{\theta}(\bar{w} + \widetilde{w}(t)) - \widetilde{\theta}(t)\bar{u}\sin\bar{\theta}
\end{equation}
