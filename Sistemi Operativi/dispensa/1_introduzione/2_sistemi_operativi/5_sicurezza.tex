\section{Sicurezza e Protezione}
Nel caso in cui più utenti devono usufruire dello stesso elaboratore l'accesso alle risorse deve essere limitato da regole imposte dal sistema operativo.

È compito delle strategie di protezione quello di fornire le specifiche dei controlli da attuare e gli strumenti per la loro applicazione.

\spacer
La sicurezza comprende tutti i meccanismi di difesa per proteggersi da attacchi interni ed esterni.

Alcuni tipi di attacco possono essere:
\begin{itemize}
    \item \textbf{Denial of Service (DoS o DDoS):} L'attacco informatico si concentra sull'esaurire deliberatamente le risorse di un sistema con l'obiettivo di renderlo incapace di erogare il servizio per cui era stato progettato
    \item \textbf{Tojan:} Programmi con una funzione legittima, per la quale l'utente esegue il programma, ed una funzione dannosa nascosta.
    \item \textbf{Worm:} Malware in grado di autoreplicarsi e distribuirsi in modo estremamente rapido.
    \item \textbf{Virus:} Porzioni di codice dannoso che si legano ad altri programmi per diffondersi.
\end{itemize}

\begin{note}
    \subsubsection*{Hacker e Cracker}
    Un hacker è semplicemente colui che sfrutta le proprie capacità informatiche per esplorare, divertirsi e apprendere senza creare danni reali.

    Mentre il cracker sfrutta le sue abilità per distruggere, ingannare o arricchirsi.
\end{note}

La funzione di sicurezza del sistema operativo si basa sull'identificazione degli utenti così da poter per determinare chi può fare cosa.