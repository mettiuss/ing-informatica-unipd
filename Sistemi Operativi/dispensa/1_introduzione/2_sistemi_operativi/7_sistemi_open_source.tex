\section{Sistemi Operativi Open Source}
Sono sistemi operativi disponibili in \textbf{formato sorgente} anziché come codice binario.

Questo approccio ha il vantaggio di aprire il sistema ad una vasta rete di programmatori, i quali sono liberi di contribuire al software aggiungendo funzionalità e risolvendo bug e falle di sicurezza.

\begin{note}
    È da notare che software opensource e free non sono sinonimi. Il software libero permette l'utilizzo, la redistribuzione e la modifica senza costo, mentre quello opensource può avere delle licenze che limitano la libertà degli sviluppatori.
\end{note}

\subsubsection*{Storia del Software Libero}
Il codice viene inizialmente concepito come libero, i primi sviluppatori dell'MIT lasciavano i loro codici sorgenti nei cassetti per lasciare che altri continuassero il lavoro.

\spacer
Durante gli anni 70 e 80 però le aziende iniziano a cercare modi per permettere l'esecuzione del loro software ai soli clienti paganti, quindi iniziano a distribuire i file compilati invece del codice sorgente.

\spacer
Per combattere questo spostamento verso il software proprietario, nel 1984 \textit{Richard Stallman} inizia a sviluppare un sistema operativo libero, compatibile e simile a Unix (il sistema operativo più diffuso al tempo), chiamato GNU (acronimo ricorsivo di \textit{\textbf{G}NU's \textbf{N}ot \textbf{U}nix!}).

\spacer
Stallman non è contrario all'idea di mettere un prezzo al proprio software, ma crede che i clienti debbano avere 4 libertà:
\begin{sitemize}
    \item \textbf{Eseguire} liberamente il programma
    \item Studiare e \textbf{modificare} il codice sorgente
    \item \textbf{Ridistribuire} o vendere il codice senza modifiche
    \item \textbf{Ridistribuire} o vendere il codice con delle modifiche
\end{sitemize}

\spacer
Nel 1985 Stallman pubblica il manifesto di GNU che suggerisce che tutto il software dovrebbe essere libero e fonda la \textit{Free Software Foundation (FSF)} con lo scopo di incoraggiare l'uso e lo sviluppo di software libero.

Stallman applica il "copyleft", ovvero l'esatto opposto del copyright, a tutto il suo lavoro che lo rende completamente libero, con l'unica condizione che debba essere ridistribuito come software libero.

Questo ideale viene poi sintetizzato nella \textit{GNU General Public License (GPL)}

\spacer
Nel 1991 il progetto GNU aveva sviluppato molti tool per lo sviluppo, ma non era ancora arrivato ad un kernel completo. Linus Torvalds, uno studente Finlandese, inizia a sviluppare un sistema operativo usando gli strumenti sviluppati dal progetto GNU.

Grazie anche all'avvento di internet, che permette a sviluppatori da tutto il mondo di scaricare il codice sorgente e di contribuire attivamente al progetto, nasce il sistema operativo \textit{Linux}.

\spacer
Dal kernel nascono moltissime distribuzioni, ogniuna con i suoi obiettivi e le sue particolarità.

\begin{note}
    \textbf{MINIX} è uno dei sistemi operativi più diffusi, anche se raramente ne sentiamo parlare. È stato sviluppato dal professor Andrew Tanenbaum come supporto didattico al suo corso universitario, con la caratteristica di essere estremamente modulare.

    Il sistema operativo è così diffuso perché intel lo utilizza per L'Intel Managment Unit, un software che viene eseguito da tutti i processori intel. Il sistema viene eseguito al livello di privilegio più elevato, con permessi pressoché illimitati sull'hardware.
\end{note}