\section{Servizi}

Alcune classi di servizi che il sistema operativo può fornire nei confronti dell'utente:
\begin{sitemize}
    \item \textbf{Interfaccia Utente:}

    Quasi tutti i sistemi operativi sono dotati di un'interfaccia utente, che può prendere molte forme diverse, la linea di comando, un'interfaccia grafica, ...

    Si può interagire con essa in molti modi, con mouse e tastiera, con uno schermo touch, ...

    \item \textbf{Esecuzione di Programmi:}

    Il sistema deve essere in grado di caricare un programma in memoria ed eseguirlo, inoltre deve essere in grado di rilevare e gestire situazioni di errore.

    \item \textbf{Operazioni I/O:}

    Il sistema operativo fornisce ai programmi dell'utente un modo per accedere ai dispositivi I/O.

    \item \textbf{Gestione del File-System:}

    I programmi utente devono essere in grado di creare, leggere, scrivere ed eliminare file e devono potersi muovere nella struttura dell'archiviazione.

    \item \textbf{Comunicazioni:}

    I programmi devono essere in grado di comunicare ad altri processi presenti sullo stesso sistema oppure ad altri sistemi connessi via rete.
    La comunicazione può essere implementata tramite memoria condivisa, oppure come scambio di messaggi

    \item \textbf{Rilevamento di Errori:}

    Il sistema operativo deve tenere sotto costante controllo CPU, memoria e dispositivi I/O per rilevare e gestire gli errori che sorgono durante l'esecuzione dei programmi utente.
\end{sitemize}

\spacer
Ci sono altri servizi che il sistema operativo deve eseguire per assicurarsi che il sistema funzioni in modo efficiente:

\begin{sitemize}
    \item \textbf{Allocazione di Risorse:}

    Quando ci sono più processi o più utenti diventa fondamentale gestire in modo efficiente le risorse hardware tra di essi, in particolare la CPU e la memoria.
    I dispositivi I/O spesso vengono gestiti in modo meno rigido utilizzando delle regole generali.

    \item \textbf{Logging:}

    Risolta essere importante registrare quali programmi e in quali quantità utilizzano le risorse del sistema.
    Queste statistiche sono preziose per gli amministratori che possono capire l'effettiva necessità di risorse.

    \item \textbf{Protezione e Sicurezza:}

    Ai possessori di informazioni su un sistema multiutente o distribuito deve essere garantito i propri dati da accessi indesiderati.

    La protezione riguarda il controllo di tutti gli accessi alle risorse di sistema.

    La sicurezza si basa sull'obbligo di autenticazione mediante password o sistema equivalente.
\end{sitemize}

\subsection{Interfaccia Utente}
Vista la sua importanza dal punto di vista dell'utente può essere utile approfondire le interfacce che permettono agli utenti di comunicare con il sistema operativo.

\subsubsection*{Command Line Interpreter (CLI)}

L'interfaccia a linea di comando permette di impartire comandi direttamente al sistema operativo.
Viene talvolta implementata dal kernel oppure attraverso programmi di sistema, inoltre può essere anche installata dall'utente, il sistema operativo può offrire più \textit{shell}.

\spacer
I comandi impartiti dall'utente possono essere eseguiti secondo due modalità:

\begin{sitemize}
    \item Se il codice del comando è relativo all'interprete viene chiamata direttamente la funzione richiesta dall'utente.

    \item Se invece il codice del comando viene implementato da un altro programma di sistema l'interprete utilizza il comando per accedere al file contenente il codice per l'esecuzione e lo inizializza.
\end{sitemize}

\begin{note}
    Spesso nei sistemi operativi non sono disponibili tutte le funzionalità tramite l'interfaccia grafica, in questi casi si può accedervi tramite l'interprete a linea di comando.

    \spacer[4pt]
    Inoltre la linea di comando risulta essere utile per l'esecuzione di comandi ripetuti in quanto è programmabile
\end{note}

\subsubsection*{Graphical User Interface (GUI)}

L'interfaccia grafica user-friendly realizza la metafora della scrivania (\textit{desktop}).

L'utente può muovere un cursore su icone che rappresentano programmi, file, cartelle e funzioni di sistema.

Questo rende semplice l'interazione con il sistema tramite mouse, tastiera e monitor.

\begin{note}
    L'interfaccia grafica nasce dal lavoro del centro di ricerca \textit{Xerox Palo Alto Research Center (PARC)}, le cui idee vennero poi riprese da Steve Jobs nella creazione del software per il primo Machintosh.

    \spacer[4pt]
    Gli sviluppatori di apple non hanno però mai avuto accesso al codice, quindi provarono ad ricrearlo partendo da una semplice dimostrazione, nella maggior parte delle situazioni riusciranno addirittura a migliorare le funzionalità del sistema della Xerox.

    \spacer[4pt]
    La storia ha un triste epilogo con la Xerox che provò a fare causa ad apple sul copyright dell'interfaccia, ma una corte statunitense porrà fine alla causa un anno dopo scagionando apple.
\end{note}
