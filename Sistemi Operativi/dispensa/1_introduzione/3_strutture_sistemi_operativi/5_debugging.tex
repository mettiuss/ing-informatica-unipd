\section{Debugging e Tuning}
Il \textbf{debugging} è l'attività di individuazione e risoluzione dei problemi, il che comprende sia la risoluzione dei bug che il performance tuning.

\spacer
Il sistema operativo aiuta nel processo con la generazione di \textbf{file di log} che danno informazioni sugli errori rilevati durante l'esecuzione.
Inoltre il sistema operativo fornisce un \textbf{core dump}, ovvero una copia della memoria impiegata dal processo al momento della terminazione anomala.

\begin{note}
    \textbf{Kernel}

    Quando avviene un errore al kernel viene chiamato \textbf{crash}. Quando questo avviene viene creato un \textbf{crash dump} che viene inserito in un'apposita sezione della memoria.

    \spacer
    È necessario di scrivere su un'area riservata di memoria perché quando si ottiene un crash lo stato del sistema non è garantito ed è difficile comprendere quali sezioni della memoria sono libere.
\end{note}

\subsection{Tuning}
Anche i problemi che condizionano le prestazioni sono considerati bachi e vanno quindi trovati e risolti.
Il performance tuning è l'insieme delle tecniche atte ad ottimizzare le prestazioni del sistema ed ad eliminare i colli di bottiglia.

\spacer
Per fare ciò ci sono varie tecniche, è possibile eseguire del codice che misuri le prestazioni del sistema e poi ne salvi i risultati su un file di log.

Oppure possono essere utilizzati degli strumenti grafici del sistema operativo, ad es. task manager di Windows.

Infine esiste anche il profiling che punta a visualizzare quali chiamate a sistema vengono utilizzare maggiormente così da ottimizzarle, quando possibile.