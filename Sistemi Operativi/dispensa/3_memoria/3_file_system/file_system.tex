\chapter{File System}

La struttura logica con la quale il sistema operativo gestisce i file prende il nome di \textit{file system}.

Essa fornisce meccanismi di registrazione, accesso e protezione ai dati e programmi degli utenti e del sistema.

\spacer
Tutti i moderni sistemi operativi utilizzano un file system \textbf{gerarchico} a cui gli utenti sono abituati.

L'organizzazione gerarchica non viene rispecchiata fisicamente nel disco, è solo una particolare astrazione dei dati.

\begin{figure}[H]
    \centering
    \includegraphics[width=0.36\linewidth]{assets/HFS.jpg}
\end{figure}

Il file system ha il compito di fornire all'utente una serie di funzioni ad alto livello che permettono di lavorare sui file mascherando i dettagli implementativi. Tutto questo deve essere svolto a prescindere dalle caratteristiche fisiche delle memorie secondarie.

\section{Directory}
Tipicamente tutte le informazioni sui file vengono conservate nella struttura \textbf{directory}, che risiede sullo stesso dispositivo dei dati.
Al suo interno viene salvato, per ogni file, un identificatore e altri attributi:
\begin{sitemize}
    \item \textbf{Nome}
    \item \textbf{Tipo}
    \item \textbf{Indirizzo}
    \item \textbf{Lunghezza attuale}
    \item \textbf{Lunghezza massima}
    \item Data \textbf{ultimo accesso}
    \item Data \textbf{ultima modifica}
    \item \textbf{ID del proprietario}
    \item Informazioni di \textbf{protezione}
\end{sitemize}

\spacer
Sulla directory è possibile fare anche delle operazioni:
\begin{sitemize}
    \item \textbf{Ricerca}
    \item \textbf{Creazione} di un file
    \item \textbf{Eliminazione} di file
    \item \textbf{Elenco} di tutti i file
    \item \textbf{Ridenominazione} di un file
    \item \textbf{Attraversamento} del file system
\end{sitemize}

\spacer
La struttura dati della directory deve essere in grado di garantire buone prestazioni e deve permettere di raggruppare i file secondo specifiche caratteristiche.

\subsubsection*{Directory Monolivello}
La struttura più semplice è quella che pone tutti i file allo stesso livello.

Tuttavia questa struttura non supporta nessun raggruppamento logico e presenta problemi nel caso di due file con lo stesso nome.
\begin{figure}[H]
    \centering
    \includegraphics[width=0.5\linewidth]{assets/directory-monolivello.jpg}
\end{figure}

\subsubsection*{Directory a Due Livelli}
Ogni utente ha la sua directory ad un livello separata.

Questa struttura ha un possibile raggruppamento (per utente) ed non presenta problemi se due utenti hanno un file con lo stesso nome, tuttavia risulta essere ancora estremamente limitata.

\begin{figure}[H]
    \centering
    \includegraphics[width=0.5\linewidth]{assets/directory-2-livelli.jpg}
\end{figure}

\subsubsection*{Directory con struttura ad albero aciclico}
Una struttura ad albero è naturale per risolvere il problema della directory, infatti essa permette una ricerca rapida, un'ottima capacità di raggruppamento e permette file con lo stesso nome in directory diverse.

\spacer
Tutte le azioni (creazione di file, creazione di directory, cancellazione di directory) avvengono a partire da una directory.

\spacer
Un altro vantaggio della struttura ad albero è la possibilità di utilizzare più nomi diversi per indicare lo stesso file.
Questo può essere implementato solo come reference (soft link) oppure duplicando il file in memoria (hard link).

Questo porta a dei problemi per l'eliminazione del file, è necessario infatti eliminare anche tutte le reference. Viene quindi tenuto un counter delle reference ed il file non viene eliminato finché il counter non arriva a 0.

\begin{figure}[H]
    \centering
    \includegraphics[width=0.42\linewidth]{assets/directory-albero-aciclico.jpg}
\end{figure}

\subsubsection*{Directory con struttura ad albero generale}
Se permettiamo anche la presenza di cicli nel grafo si ottengono dei problemi per l'attraversamento, vogliamo assicurarci che non ci siano file che vengono visitati due volte.

Per fare questo in fase di attraversamento è necessario marcare tutti i file visitati.

\spacer
Un'altra complicazione si trova nel fatto che un blocco può avere counter delle reference diverso da 0 pur non avendo nessun blocco che lo punta, per questo viene utilizzato un algoritmo di \textit{garbage collection} il quale attraversa l'albero e poi elimina tutti i file non contrassegnati come visitati.

\begin{figure}[H]
    \centering
    \includegraphics[width=0.42\linewidth]{assets/directory-grafo.jpg}
\end{figure}

\subsubsection*{Stabilire se un grafo è aciclico}
Per stabilire se un grafo è aciclico possiamo sfruttare il fatto che un grafo si dice aciclico se ha (almeno) un ordinamento topologico. Ovvero se è possibile assegnare ai nodi degli indici in modo che $$\forall \text{ ramo orientato } e = (u, v) \in G, index(u) < index(v)$$

Algoritmo per trovare un ordinamento topologico:
\begin{sitemize}
    \item Assegniamo (a caso) numeri crescenti ai vertici privi di rami entranti che non siano ancora numerati (se non ce ne sono, il grafo ha almeno un ciclo)
    \item Eliminiamo dal grafo i rami uscenti dai vertici a cui abbiamo assegnato un numero
    \item Ripetiamo finché ci sono vertici non numerati
\end{sitemize}

\begin{figure}[H]
    \centering
    \includegraphics[width=0.75\linewidth]{assets/algoritmo-ordinamento-topologico.jpg}
\end{figure}

\subsubsection*{Implementazione}
La struttura dati può essere implementata tramite una lista o una lista concatenata, ma questa soluzioni ha importanti problemi sulle prestazioni.

Si preferisce utilizzare una tabella hash con chiave: nome del file, valore: puntatore all'identificatore del file. Essa porta a prestazioni migliori, anche se può avere necessità di rehash.

\subsection{Montaggio}
Per poter accedere ad un file system è necessario montarlo, ovvero attaccarne la struttura dati ad un punto di montaggio.

\spacer
La procedura richiede di fornire al sistema operativo il nome del dispositivo e il punto di montaggio, ovvero una directory a cui viene agganciato il file system.

Il punto di montaggio non deve essere una directory vuota, ma questo è preferibile in quanto i file in essa contenuti non saranno visibili fino all'operazione di unmount.

\spacer
I sistemi macOS e Windows rilevano tutti i dispositivi all'avvio e montano automaticamente tutti i loro file system.

Nei sistemi Unix è invece necessario montare i file system manualmente dopo l'avvio.

I sistemi Linux sono simili a quelli unix, ma forniscono anche un file di configurazione che permette di descrivere quali dispositivi montare in modo automatico.

\subsection{Esempi}
\subsubsection{Unix/Linux}
Viene costituito un grafo generale di directory, che ospita tutti i file e le directory.
A ciascun utente viene associata una directory come sottodirectory di \texttt{/usr}.

\spacer
Ciascun file viene identificato univocamente da un \textit{pathname}, questo significa che tutti i file e sottodirectory devono avere nomi distinti.

\spacer
Ogni utente che interagisce con il file system ha un proprio contesto, visualizzabile con \texttt{pwd} che può essere modificato grazie al comando \texttt{cd}.

\spacer
Esistono due tipi di link in linux, l'hard link che crea un nuovo elemento nella directory che punta allo stesso file fisico e il soft link che punta solamente ad un'altro elemento della directory.

Alla cancellazione dell'elemento gli hard link rimangono, mentre i soft link vengono tutti eliminati.

\spacer
Il comando \texttt{chmod} permette di modificare il livello di protezione assegnato al file.
È possibile utilizzare il comando per assegnare dei permessi sovrascrivendo quelli presenti o per modificare quelli già esistenti.

\begin{note}
    A partire da Fedora 33 è stato selezionato Btrfs come file system di default, esso fornisce dei servizi di pooling (risorse pronte per essere utilizzate), snapshots (una copia read only generata in O(1)) e checksum.
\end{note}

\section{Implementazioni}
Il file system è stratificato, ogni livello si serve di funzioni dei livelli inferiori per realizzare funzioni utilizzate dai livelli superiori.

\begin{sitemize}
    \item \textbf{Dispositivi Hardware}
    \item \textbf{Controllo I/O:} Forniscono comandi che permettono di leggere specifiche locazioni di memoria del disco (leggi dal disco 1, cilindro 72, traccia 2, settore 10 nella locazione di memoria 1060) e li traducono in sequenze di bit che fanno spostare la testina alla locazione.

    \item \textbf{File System di Base:} Gestisce buffer e cache del dispositivo, fornisce comandi di più alto livello (leggi il blocco 123)
    \item \textbf{Modulo di organizzazione dei file:} Traduce indirizzi logici in indirizzi fisici
    \item \textbf{File System logico:} Gestisce le strutture dati del file system.
\end{sitemize}

\subsubsection*{Strutture Dati su Disco}
\begin{sitemize}
    \item \textbf{Blocco di controllo di avviamento:} O partizione di boot, contiene le informazioni necessarie all'avvio del disco, normalmente inizia dal primo blocco del disco.
    \item \textbf{Blocco di controllo dei volumi:} Contengono dettagli riguardo la partizione quali: numero dei blocchi (e dimensione), contatore blocchi liberi (e i loro puntatori)
    \item \textbf{Struttura delle directory}
    \item \textbf{Blocchi di controlli dei file:} Contengono i dettagli di ogni file quali: permessi, date (creazione, modifica), owner, dimensione, puntatore.

    NTFS utilizza una struttura dati in stile DB relazionale per queste informazioni.
\end{sitemize}

\subsubsection*{Strutture Dati su Memoria}
\begin{sitemize}
    \item \textbf{Tabella di montaggio:} Contiene informazioni su ogni volume montato
    \item \textbf{Struttura delle directory:} Una cache della struttura dati fisica.
    \item \textbf{Tabella dei file aperti}
    \item \textbf{Tabella dei file aperti per ciascun processo}
\end{sitemize}

\subsubsection*{ISO 9660}
Utilizzato dalle CD-ROM

\subsubsection*{UFS}
Unix File System, creato su il File System Berkley Fast (FFS)

\subsubsection*{Windows}
Supporta FAT, FAT32, FAT64, NTFS

\subsubsection*{Linux}
Fornisce ext2, ext3, ext4 (extended file system), ma supporta altri 40 tipi diversi.

\subsubsection*{GoogleFS}
Detto anche BigFiles è un file system ottimizzato per lo storage di file di grandi dimensioni (>100GB) con poche modifiche o eliminazioni.

\subsubsection*{OracleASM}
Gestisce file, directory, volumi grazie a direttive SQL.

\subsection{File System Virtuali}
Fornisce un'interfaccia comune per l'accesso a dispositivi con file system differenti.

\begin{figure}[H]
    \centering
    \includegraphics[width=0.5\linewidth]{assets/virtual-file-system.png}
\end{figure}

\subsection{NetWork File System}
Permette di unificare file system indipendenti di computer indipendenti nella rete, permette a qualsiasi utente di accedere al file system di qualsiasi altro computer.

In particolare rende possibile montare una directory remota specificando la posizione del calcolatore e quella del montaggio.

\begin{figure}[H]
    \centering
    \includegraphics[width=0.5\linewidth]{assets/NFS.png}
\end{figure}

\subsubsection{FTP}
È possibile condividere file system anche attraverso la rete, il \textit{File Transfer Protocol (FTP)} permette di visualizzare ed accedere al file system di un altro calcolatore.

Questo è un modello client-server molti-a-molti, un server può gestire le richieste di più utenti ed un utente può accedere a molti server.
In questo caso per gestire l'autenticazione vengono utilizzate delle chiavi di cifratura, che introducono un'altra serie di complicazioni.

