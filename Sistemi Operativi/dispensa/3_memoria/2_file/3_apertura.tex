\section{Apertura di un File}
Prima di un qualsiasi accesso ai dati del file occorre trovare l'indirizzo fisico dei dati, ovvero è necessario "aprire" il file.

Il sistema operativo tiene poi in memoria una tabella dei file aperti che permette di accedere rapidamente ai file in uso.

Le system call che gestiscono questa funzione sono:
\begin{sitemize}
    \item \textbf{open(F):} ricerca il file nella directory del disco e copia il puntatore ai dati nella tabella dei file.

    \item \textbf{close(F):} copia il contenuto residente nella memoria principale alla memoria secondaria e rimuove il file dalla tabella.
\end{sitemize}

\subsection{Sistemi multiprogrammati}
Nei sistemi multiprogrammati la gestione dei file aperti è più complicata in quanto file possono essere aperti da più processi.

Il sistema operativo mantiene due tipi di tabelle dei file, una a livello del sistema e un'altra per ogni processo.

\subsubsection{Tabella di Sistema}
La tabella di sistema mantiene riferimenti a tutti i file aperti nel sistema, per ogniuno essa salva:

\begin{sitemize}
    \item \textbf{Posizione} del file nel disco
    \item \textbf{Dimensione} del file
    \item Date di \textbf{ultimo accesso} e \textbf{ultima modifica}
    \item \textbf{Contatore} di aperture
\end{sitemize}

\spacer
In particolare il \textbf{contatore di aperture} permette di assicurarsi di chiudere il file al livello del sistema operativo solamente quando tutti i processi hanno smesso di utilizzarlo.

\subsubsection{Tabella del Processo}
La tabella associata al processo mantiene riferimenti a tutti i file aperti dal processo, per ogniuno essa salva:

\begin{sitemize}
    \item Puntatore alla posizione corrente nel file
\end{sitemize}

\spacer
Inoltre alcuni sistemi operativi forniscono un \textbf{lock} per mediare sull'accesso dei file.
Esso può essere \textbf{obbligatorio} oppure \textbf{consigliato}.

Nei sistemi in cui il lock è obbligatorio (Windows) non è possibile accedere ai file che sono già aperti da un'altro processo, invece nei sistemi con un lock consigliato (Unix) viene passato allo sviluppatore il compito di mantenere l'integrità dei dati.

\subsection{Condivisione}
Nei sistemi multiutente è utile che i file vengano condivisi tra più utenti, tuttavia questo deve avvenire seguendo un preciso schema di protezione.

\spacer
Il modello più diffuso prevede un \textbf{propretario} di ogni file, il suo creatore, che può cambiare gli attributi tra i quali si trova il gruppo di utenti che possono accedere al file.

\subsubsection{Coerenza}
Quando più utenti possono apportare modifiche allo stesso file si possono verificare delle situazioni simili alle race condition che avvengono nella sincronizzazione fra processi.

Spesso si impone che le scritture di un utente risultino immediatamente visibili agli altri utenti, il file ha quindi un'unica immagine a cui tutti gli utenti accedono.
