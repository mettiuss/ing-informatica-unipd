\section{Tipo di Dati Astratto}
Un file è un tipo di dati astratto su cui sono definite le operazioni di:
\begin{sitemize}
    \item \textbf{Creazione}

    Reperisce dello spazio per memorizzare il file nella memoria e crea un nuovo elemento nella directory.

    \item \textbf{Scrittura}

    Viene fatta una \textit{system call} che specifica il file su cui scrivere e i dati da scrivere.
    Il sistema operativo mantiene un puntatore di scrittura per continuare una scrittura sequenziale.

    \item \textbf{Lettura}

    \textit{system call} con riferimento al file e indirizzo di memoria principale su cui scrivere i dati. Anche in questo caso il sistema operativo mantiene un puntatore di lettura per le letture sequenziali.

    \item \textbf{Posizionamento del File}

    Riposiziona il puntatore di lettura/scrittura ad un diverso punto del file.

    \item \textbf{Cancellazione}

    Si rilascia lo spazio allocato al file e lo si rimuove dalla directory. (I bit non vengono azzerati quindi i dati del file possono essere ancora letti)

    \item \textbf{Troncamento}

    Elimina il contenuto del file, ma ne mantiene tutti gli attributi

    \item \textbf{Impostazione degli attributi}

    Reperimento/aggiornamento delle informazioni del file nel relativo elemento della directory
\end{sitemize}

Da queste operazioni elementari è possibile poi costruirne di più complesse facendone una combinazione.

