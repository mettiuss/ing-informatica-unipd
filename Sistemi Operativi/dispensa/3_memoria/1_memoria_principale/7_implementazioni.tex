\section{Implementazioni}
\subsection{Linux}
Per quanto riguarda la \textbf{Memoria del Kernel} linux utilizza il sistema Buddy, un allocatore potenza di 2.
Invece per le strutture dati, a partire dal kernel 2.4, viene utilizzato l'allocatore SLOB o SLUB.
\begin{sitemize}
    \item \textbf{SLOB:} Utilizzato per sistemi con poca memoria, mantiene 3 liste, per oggetti grandi, medi e piccoli.
    \item \textbf{SLUB:} Allocatore a lastre ottimizzata per sistemi multicore.
\end{sitemize}

\spacer
Invece la \textbf{Memoria Utente} viene gestita tramite paginazione su richiesta.
La politica di sostituzione è detta Clock: vengono mantenute due liste, \texttt{active\_list} e \texttt{inactive\_list}, la prima che contiene le pagine in uso e la seconda contiene pagine non utilizzate di frequente che possono essere rimosse.

\spacer
Ogni pagina ha un bit di accesso.
\begin{sitemize}
    \item Quando una pagina viene \textbf{allocata} o \textbf{riferita} il bit viene impostato su 1 e viene inserita in coda a \texttt{active\_list}.
    \item La pagina utilizzata meno di recente si sposta in testa alla lista \texttt{active\_list}, dalla quale viene \textbf{spostata} verso la lista \texttt{inactive\_list} per mantenere le due liste bilanciate.
    \item Inoltre i bit di accesso vengono resettati periodicamente, riportando il sistema ad uno stato iniziale con tutti i processi in \texttt{inactive\_list}.
    \item Il deamon \texttt{kswapd} si risveglia periodicamente e se la memoria non è sufficiente \textbf{termina} dei programmi dalla \texttt{inactive\_list}.
\end{sitemize}

\subsection{Windows}
Per quanto riguarda la \textbf{Memoria del Kernel} Windows scrive i dati su un page file, ma quando questo non è possibile i dati vengono scritti su una pool di memoria non paginata. (ad es. quando Windows non può fare page fault perché sta gestendo un page fault.)

\spacer
Invece la \textbf{Memoria Utente} viene gestita tramite paginazione con clustering, quindi quando avviene un page fault viene caricato non solo la pagina richiesta, ma anche le pagine adiacenti.

Ad un processo viene assegnato una dimensione minima e massima del working set, il minimo è il numero di pagine garantite dal sistema operativo, il numero massimo è il limite oltre il quale il processo non può allocare.

Quando l'occupazione della memoria scende si effettua una riassegnazione dei working set aumentando i tetti massimi per processo.

\spacer
Windows fa inoltre uso di compressione delle pagine più vecchie, prima di ricorrere alla memoria virtuale. Quando poi i dati tornano ad essere utilizzati devono prima essere decompressi.
