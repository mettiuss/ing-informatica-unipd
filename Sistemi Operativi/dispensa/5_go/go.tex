\chapter{Go}
Durante il corso viene consigliato lo studio del linguaggio di programmazione Go in quanto rende semplice la creazione di programmi che supportano concorrenza.

\section{Download}
Pagina di Download: \href{https://go.dev/dl/}{https://go.dev/dl/}

\section{Sintassi}

\subsection{Packages}
I \textit{package} sono alla base di ogni programma in Go, non è infatti possibile creare un programma senza il package main.

\spacer
La struttura del progetto dovrebbe essere:
\begin{minted}{bash}
progetto/
    utils/
        sum.go
    main.go
    go.mod
\end{minted}

\spacer
Dove nel file \texttt{go.mod} inseriamo una definizione del progetto:
\texttt{go.mod}:
\begin{minted}{go}
module progetto
go 1.22
\end{minted}

\spacer
Il package main deve contenere una funzione main dalla quale l'esecuzione del programma avrà inizio. Nel file \texttt{main.go} inseriamo quindi:
\begin{minted}{go}
package main

import (
    "fmt"
    "progetto/utils"
)

func main() {
    fmt.Println(utils.Sum(2, 3))
}
\end{minted}

Dal file \texttt{sum.go} vengono esportate tutte le variabili e funzioni che iniziano con la lettera maiuscola, le altre rimangono locali al file.

\subsection{if/switch}
Gli \texttt{if} in go hanno una sintassi simile agli altri linguaggi, viene però aggiunta la possibilità di inserire un piccolo statement prima della condizione.
\begin{minted}{go}
if v := math.Pow(x, n); v < lim {
    return v
}
\end{minted}

Gli switch possono avere una o più condizioni
\begin{minted}{go} 
switch condition {
	case x:
		...
	case y, z:
		...
	default:
		...
}
\end{minted}

\subsection{Cicli}
In Go esiste un solo ciclo, il ciclo for, il quale ha una sintassi che può essere adattata per diventare un while ed un foreach.
\begin{minted}{go}
for i:=1; i<10; i++{ // for loop
    ...
}

for num < 8 { // while loop
    ...
}

for i := range sl { //foreach
    fmt.Println(i, sl[i])
}
\end{minted}

\subsection{Funzioni}
In Go le funzioni hanno una sintassi semplice, vediamo la stessa funzione in una forma estesa e poi in una più compatta

\begin{minted}{go}
func calculate(x int, y int) (int, int) {
    var sum = x + y
    min := x - y
    return sum, min
}
\end{minted}

\begin{minted}{go}
func calculate(x, y int) (a, b int) {
    a = x + y
    b = x - y
    return //naked return
}
\end{minted}

\subsection{struct}
Sono una collezione di campi

\begin{minted}{go}
// Definizione
type Books struct{
    title string
    author string
}

// Costruttore (di default)
b := Books {"titolo", "autore"}

// Accesso ai membri
b.title
b.author
\end{minted}

\subsection{defer}
Permette di eseguire uno o più statement (in questo caso sono inseriti in uno stack LIFO) al termine della funzione chiamante.

\begin{minted}{go}
defer fmt.Println("primo")
fmt.Println("secondo")
// output => "secondo", "primo"
\end{minted}

\section{Concorrenza}
Gli strumenti principali per la concorrenza in Go sono le goroutine, ovvero dei thread che vengono avviati da un'altro thread mediante la keyword \texttt{go}.

\begin{minted}{go}
func foo(){
    fmt.Println("foo")
}

func main() {
    go foo() // main avvia foo in un altro thread
    fmt.Println("main")
}
\end{minted}

\begin{note}
    In questo caso spesso "foo" non viene stampato in quanto termina prima l'esecuzione della funzione main(), che di conseguenza termina il programma e tutte le sue subroutine.
\end{note}

\subsection{Channels}
I channels sono degli strumenti che permettono la comunicazione tra due threads in Go.

\spacer
Vengono creati utilizzando la funzione \texttt{make(chan int, 100)}, dove il primo argomento specifica il tipo e il secondo il buffer.

Si utilizza -> per inviare e ricevere i dati.

\spacer
Un thread si blocca fino a che l'istruzione di inviare o ricevere un elemento non si è completata.

Da un channel chiuso si ottiene l'elemento zero della classe.

\subsection{Select}
Statement che permette di attendere il primo thread a completare l'esecuzione.
La keyword default permette di continuare l'esecuzione quando non sono presenti elementi nei channel, default -> attende una certa quantità di tempo prima di proseguire.

\begin{minted}{go}
func main(){
    c1 := make(chan int)
    c2 := make(chan int)
    ...
    select{
        case x := <- c1:
            ...
        case x := <- c2:
            ...
        case <-time.After(time.Duration(rand.Intn(5))*time.Second)
            ...
    }
}
\end{minted}

\subsection{WaitGroup}
Per attendere che un gruppo di thread completi l'esecuzione, si può utilizzare WaitGroup.

\begin{sitemize}
    \item \texttt{waitgroup.Add(n)} aggiunge all'oggetto un numero n di thread da aspettare
    \item \texttt{waitgroup.Done()} comunica che il thread ha completato l'esecuzione
    \item \texttt{waitgroup.Wait()} attende che tutti i thread abbiano chiamato Done()
\end{sitemize}

\begin{minted}{go}
func worker(id int) {
    fmt.Printf("Worker %d starting\n", id)

    time.Sleep(time.Second)
    fmt.Printf("Worker %d done\n", id)
}

func main() {
    var wg sync.WaitGroup

    for i := 1; i <= 5; i++ {
        wg.Add(1)

        go func() {
            defer wg.Done()
            worker(i)
        }()
    }
    wg.Wait()
}
\end{minted}

\begin{note}
    L'oggetto waitgroup deve essere passato come puntatore
\end{note}
