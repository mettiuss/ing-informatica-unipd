\section{Race Condition}
Quando più processi accedono in concorrenza e modificano i dati condivisi l'esito finale dell'esecuzione è indeterminato. Dipende dall'ordine in cui le loro istruzioni vengono eseguite.
Questa situazione viene chiamata \textbf{race condition} e va gestita per evitare di ottenere risultati erronei.

\subsubsection*{Esempio}

\spacer
\begin{minipage}{0.45\textwidth}
    \begin{minted}{java}
// Thread 1
for (int i = 0; i < 5; i++) {
    x++;
}
\end{minted}
\end{minipage}
\hfill
\begin{minipage}{0.45\textwidth}
    \begin{minted}{java}
// Thread 2
for (int j = 0; j < 5; j++) {
    x--;
}
\end{minted}
\end{minipage}
\spacer

In questo caso, se i due thread vengono eseguiti concorre mente, il valore finale di $x$ non è noto, potrebbe variare tra $-5$ e $+5$.

\subsubsection*{Operazioni Atomiche}

Per risolvere il problema della \textit{race condition} è necessario svolgere le operazioni di lettura e modifica delle variabili condivise senza interruzioni. Queste operazioni sono definite \textbf{atomiche}.
