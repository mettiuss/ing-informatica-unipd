\section{Lock}
In generale qualsiasi soluzione al problema della sezione critica utilizza il \textbf{lock}. Per accedere ad una sezione critica il processo deve acquisire un lock, che restituisce poi all'uscita dalla sezione critica.

\begin{minted}{java}
{
    //acquisisce il lock
    sezione critica
    //restituisce il lock
    sezione non critica
}
\end{minted}

\subsubsection*{Supporto Hardware}
Molte architetture forniscono il supporto hardware per implementare efficacemente i lock, inoltre alcune forniscono anche direttamente delle istruzioni atomiche.

\subsubsection*{Supporto Sistema Operativo}
Dato che le implementazioni hardware sono spesso inaccessibili ai programmatori, il sistema operativo fornisce degli strumenti per risolvere il problema.

Il più semplice è il \textbf{lock mutex}, permette di proteggere le sezioni critiche aggiungendo \texttt{acquire()} prima del segmento e \texttt{release()} al termine
\begin{minted}{java}
acquire() {
    while(!available) {}; // busy wait
    available = false;
}

release() {
    available = true;
}
\end{minted}
