\section{Implementazione: Linux}
Prima della versione 2.6 (2004) Linux utilizzava un kernel senza diritto di prelazione. Per ottenere delle sezioni critiche venivano disabilitati momentaneamente gli interrupt.

\spacer
Ad oggi Linux fornisce diversi strumenti per la sincronizzazione:
\begin{itemize}
    \item \textbf{Lock Mutex:} Una variabile booleana che indica se il lock è disponibile, modificata tramite chiamate a sistema.
          \begin{minted}{cpp}
#include <pthread.h>
/* create and initialize the mutex lock */ 
pthread mutex t mutex;
pthread mutex init (&mutex, NULL);

pthread mutex lock (&mutex); /* acquire the mutex lock */ 
/* critical section */
pthread mutex unlock (&mutex); /* release the mutex lock */ 
    \end{minted}
    \item \textbf{Spinlock:} Un ciclo di busy waiting che verifica periodicamente se il lock è libero. È particolarmente inefficiente per i sistemi single core, in quanto altri processi non vengono eseguiti e quindi la risorsa attesa non viene sbloccata. Mentre può fornire alcuni vantaggi su sistemi multicore in quanto il processo non esce dall'esecuzione.
    \item \textbf{Semafori:} Ne esistono di due tipi, con e senza nome. I semafori con nome sono condivisi da tutti i processi imparentati, mentre quelli senza nome possono essere utilizzati solo nel contesto del processo.
          \begin{minted}{cpp}
#include <semaphore.h>
/* Create the semaphore and initialize it to 1 */
sem_t *sem;
sem = sem open ("SEM", O CREAT, 0666, 1);

sem_wait (sem); /* acquire the semaphore */
/* critical section */
sem-post (sem); /* release the semaphore */ 
            \end{minted}
    \item \textbf{Interi atomici:} Un tipo le cui operazioni sono tutte definite in modo atomico. Essi non risentono dell'overhead introdotto dai tradizionali meccanismi di lock.
          \begin{minted}{cpp}
atomic_t counter;
atomic_set (&counter, 5); /* counter = 5 */
atomic_add (10, &counter); /* counter = 15 */
atomic_sub (4, &counter); /* counter = 11 */
atomic_inc (&counter); /* counter = 12 */
int value = atomic_read (&counter); /* value = 12 */
    \end{minted}

    \item \textbf{Variabili condition:} Sono le variabili utilizzate per la realizzazione di monitor, forniscono un wait e un signal che gestiscono in autonomia un lock.

          Il wait acquisice periodicamente il lock, verifica la condition e poi libera il lock.

\end{itemize}

